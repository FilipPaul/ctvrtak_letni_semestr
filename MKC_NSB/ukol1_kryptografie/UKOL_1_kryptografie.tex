%%PREAMBLE %%%%%%%%%%%%%%%%%%%%%%%%%%%%
\documentclass[10pt, a4paper]{article}% size of txt = 10pt
\usepackage[top= 2cm,
			bottom = 2cm,
			left = 1.7cm,
			right = 1.7cm,
			footskip = 0.5cm,
			headsep = 0cm,
			headheight = 0cm
					]{geometry}
\usepackage{amsmath} % math packages
\usepackage{amsfonts}% math packages
\usepackage{amssymb} % math packages
\usepackage{graphicx} %package for including graphics
\usepackage{array}
\usepackage[thinlines]{easytable}
\usepackage{float}
\usepackage[section]{placeins}
\usepackage[hidelinks]{hyperref}
\usepackage[shortlabels]{enumitem}
\usepackage{svg}
\usepackage{bigstrut}
\usepackage{wrapfig,lipsum,booktabs}
\usepackage{subcaption}
\usepackage{xfrac}
\usepackage{pdfpages}

\usepackage{listings}
\usepackage{color} %red, green, blue, yellow, cyan, magenta, black, white
\definecolor{mygreen}{RGB}{28,172,0} % color values Red, Green, Blue
\definecolor{mylilas}{RGB}{170,55,241}


%date format
\def\mydate{\leavevmode\hbox{\twodigits\day.\twodigits\month.\the\year}}
\def\twodigits#1{\ifnum#1<10 0\fi\the#1}

\usepackage{indentfirst}
\setlength{\parindent}{1cm}

\makeatletter
\newcommand{\thickhline}{%
    \noalign {\ifnum 0=`}\fi \hrule height 2pt
    \futurelet \reserved@a \@xhline
}
\newcolumntype{"}{@{\hskip\tabcolsep\vrule width 2pt\hskip\tabcolsep}}
\makeatother
\newcolumntype{?}{!{\vrule width 2pt}}
%%DOC ENVIROMENT%%%%%%%%%%%%%%%%%%%%%%%
\begin{document}
%Title 
\begin{flushleft}%% left justification
	\textbf{\Large{MKC-NBS: Úkol č. 1}}\hfill Filip Paul\\
	\large{Kryptografie \hfill\mydate}
\end{flushleft}
\section*{\large{\textbf{Bloková šifra v režimu CBC:}}}
	\begin{enumerate}
		\item \textbf{Zadání:}\\
			Mějme zprávu Z = (13, 4, 9), kde jednotlivá čísla jsou bloky zprávy. Tuto zprávu zašifrujte v režimu
			CBC pro inicializační vektor IV = 6. Vypočítaný kryptogram pro kontrolu dešifrujte. Šifrování E a
			dešifrování D je dáno substitucemi podle tabulky 1. K provedení operací XOR si dekadická čísla
			převeďte na čtyřbitová čísla. Pro daný provozní režim nakreslete diagramy podle první přednáškové
			prezentace (snímek č. 21), přičemž v datových blocích schématu uveďte dekadicky i binárně
			hodnotu příslušného vstupu, či výstupu.

			% Table generated by Excel2LaTeX from sheet 'List1'
			\begin{table}[htbp]
				\centering
				\caption{Šifrovací substituce y = E(x,K)}
				\begin{tabular}{?c?c|c|c|c|c|c|c|c|c|c|c|c|c|c|c|c?}
				\thickhline
				X  & 0  & 1  & 2  & 3  & 4  & 5  & 6  & 7  & 8  & 9  & 10 & 11 & 12 & 13 & 14 & 15 \bigstrut\\
				\thickhline
				Y  & 4  & 10 & 9  & 2  & 13 & 8  & 0  & 14 & 6  & 11 & 1  & 12 & 7  & 15 & 5  & 3 \bigstrut\\
				\thickhline
				\end{tabular}%
				\label{tab:sifrTab}%
			\end{table}%

		\item \textbf{Vypracování:}\\
 	
	\end{enumerate}

	\section*{\large{\textbf{Výpočet pečeti HMAC}}}
		\begin{enumerate}
			\item \textbf{Zadání:}\\
				Mějme zprávu Z = (13, 4, 9), kde jednotlivá čísla jsou bloky zprávy. Pro tuto zprávu vypočítejte
				technikou HMAC pečeť P. Pečetící klíč K = 7, konstanta $C_1$ = 13 a $C_2$ = 8. K provedení operací
				XOR si dekadická čísla převeďte na čtyřbitová čísla. Hešovací funkce H je definována následovně:\\\\
				$h = \left( \sum_{i = 1}^{t}  a^i \cdot v_i \right) mod 17$\\\\
				kde hešovací konstanta a = 11, $v_i$ je i-tý blok hešovaného vstupu a t počet bloků na vstupu. V prvém
				hešování tedy bude t = 4, protože první blok je výsledek xorování klíče a konstanty a další bloky
				jsou bloky zprávy. Ve druhém hešování bude t = 2. Pečetí P je výstup z druhého hešování, tj. P = $h_2$.
				\item \textbf{Vypracování:}\\
		\end{enumerate}

		\section*{\large{\textbf{RSA podpis}}}
		\begin{enumerate}
			\item \textbf{Zadání:}\\
				Byla Vám doručena zpráva Z = (13, 4, 9), jejíž RSA podpis DS = 5. Ověřte, zda je tato zpráva
				autentická. Znáte veřejný ověřovací klíč udávaného autora VK = 3, jeho modulus n = 33 a víte, že
				byla použita hešovací funkce H ze 2. příkladu.
			\item \textbf{Vypracování:}\\

		\end{enumerate}
\end{document}

%\[f(x)= (x+2)^2 - \frac{9\cdot 2\pi}{26}\] %%mathematic equatation in display style mode
%%optional:
%	\begin{align} %%this alignes all charakters after & if *is removed equations will be numbered
%	\hspace{5cm}  
%		 x &= a_2 x^2 +_1 x + a_0 \\
% 		x &=x^2 \nonumber		%no number will not add number to eq
%	\end{align}