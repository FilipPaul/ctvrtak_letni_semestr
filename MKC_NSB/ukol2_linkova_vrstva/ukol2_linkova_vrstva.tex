%%PREAMBLE %%%%%%%%%%%%%%%%%%%%%%%%%%%%
\documentclass[10pt, a4paper]{article}% size of txt = 10pt
\usepackage[top= 2cm,
			bottom = 2cm,
			left = 1.7cm,
			right = 1.7cm,
			footskip = 0.5cm,
			headsep = 0cm,
			headheight = 0cm
					]{geometry}
\usepackage{amsmath} % math packages
\usepackage{amsfonts}% math packages
\usepackage{amssymb} % math packages
\usepackage{graphicx} %package for including graphics
\usepackage{array}
\usepackage[thinlines]{easytable}
\usepackage{float}
\usepackage[section]{placeins}
\usepackage[hidelinks]{hyperref}
\usepackage[shortlabels]{enumitem}
\usepackage{svg}
\usepackage{bigstrut}
\usepackage{wrapfig,lipsum,booktabs}
\usepackage{subcaption}
\usepackage{xfrac}
\usepackage{pdfpages}
\usepackage{listings}
\usepackage{xcolor}


\usepackage{listings}
\usepackage{color} %red, green, blue, yellow, cyan, magenta, black, white
\definecolor{mygreen}{RGB}{28,172,0} % color values Red, Green, Blue
\definecolor{mylilas}{RGB}{170,55,241}

\definecolor{codegreen}{rgb}{0,0.6,0}
\definecolor{codegray}{rgb}{0.5,0.5,0.5}
\definecolor{codepurple}{rgb}{0.58,0,0.82}
\definecolor{backcolour}{rgb}{1,1,1}

\lstdefinestyle{mystyle}{
    backgroundcolor=\color{backcolour},   
    commentstyle=\color{codegreen},
    keywordstyle=\color{magenta},
    numberstyle=\tiny\color{codegray},
    stringstyle=\color{codepurple},
    basicstyle=\ttfamily\footnotesize,
    breakatwhitespace=false,         
    breaklines=true,                 
    captionpos=b,                    
    keepspaces=true,                 
    numbers=left,                    
    numbersep=5pt,                  
    showspaces=false,                
    showstringspaces=false,
    showtabs=false,                  
    tabsize=2
}
\lstset{style=mystyle}


%date format
\def\mydate{\leavevmode\hbox{\twodigits\day.\twodigits\month.\the\year}}
\def\twodigits#1{\ifnum#1<10 0\fi\the#1}

\usepackage{indentfirst}
\setlength{\parindent}{1cm}

\makeatletter
\newcommand{\thickhline}{%
    \noalign {\ifnum 0=`}\fi \hrule height 2pt
    \futurelet \reserved@a \@xhline
}
\newcolumntype{"}{@{\hskip\tabcolsep\vrule width 2pt\hskip\tabcolsep}}
\makeatother
\newcolumntype{?}{!{\vrule width 2pt}}
%%DOC ENVIROMENT%%%%%%%%%%%%%%%%%%%%%%%
\begin{document}
%Title 
\begin{flushleft}%% left justification
	\textbf{\Large{MKC-NBS: Úkol č. 1}}\hfill Filip Paul\\
	\large{Kryptografie \hfill\mydate}
\end{flushleft}
\section*{\large{\textbf{Rámec Ethernet II:}}}
	\begin{enumerate}
		\item \textbf{Zadání:}\\
		Níže jsou uvedeny bajty dvou rámců typu Ethernet II, tak jak byly zachyceny programem 
		Wireshark. Tento program ve svých výpisech neuvádí návěští rámců ani jejich kontrolní součet, 
		takže tato pole v zadání nehledejte.

		\begin{minipage}{0.49\textwidth}
			\textbf{1. rámec:}\\
			01 00 5e 00 00 12 00 00 5e 00 01 01 08 00 45 c0\\
			00 28 00 00 00 00 ff 70 17 f0 c0 a8 01 fb e0 00\\
			00 12 21 01 c8 01 00 01 54 55 c0 a8 01 fe 00 00\\
			00 00 00 00 00 00 00 00 00 00 00 00
			
		\end{minipage}
		\begin{minipage}{0.49\textwidth}
			\textbf{2. rámec:}\\
			ff ff ff ff ff ff f0 f3 36 af f4 54 08 06 00 01 \\
			08 00 06 04 00 01 00 07 0d af f4 54 18 a6 ac 01 \\
			00 00 00 00 00 00 18 a6 ad 9f 06 01 04 00 00 00 \\
			00 02 01 00 03 02 00 00 05 01 03 01
		\end{minipage}



		\item \textbf{Vypracování:}\\
		Na následujícím obrázku jsou vyznačeny stavy v decimální i binární podobě pro každý "stupeň"
		blokové šifry. Tyto hodnoty byly vypočítány pomocí python scriptu přiloženého na konci tohoto pdf souboru.
		Nicméně pro lepší zobrazení scriptu můžete využít link na můj github repozitář $\rightarrow$ 
		\href{https://github.com/FilipPaul/ctvrtak_letni_semestr/blob/main/MKC_NSB/ukol1_kryptografie/README.md}{\color{blue} PYTHON CBC.py}
 	
	\end{enumerate}
	%\lstinputlisting[language=python]{HMAC.py}
\end{document}

%\[f(x)= (x+2)^2 - \frac{9\cdot 2\pi}{26}\] %%mathematic equatation in display style mode
%%optional:
%	\begin{align} %%this alignes all charakters after & if *is removed equations will be numbered
%	\hspace{5cm}  
%		 x &= a_2 x^2 +_1 x + a_0 \\
% 		x &=x^2 \nonumber		%no number will not add number to eq
%	\end{align}