\documentclass[10pt, a4paper]{article}% size of txt = 10pt
\usepackage[top= 2cm,
			bottom = 2cm,
			left = 1.7cm,
			right = 1.7cm,
			footskip = 0.5cm,
			headsep = 0cm,
			headheight = 0cm
					]{geometry}
\usepackage{amsmath} % math packages
\usepackage{amsfonts}% math packages
\usepackage{amssymb} % math packages
\usepackage{graphicx} %package for including graphics
\usepackage{array}
\usepackage[thinlines]{easytable}
\usepackage{float}
\usepackage[section]{placeins}
\usepackage[hidelinks]{hyperref}
\usepackage[shortlabels]{enumitem}
\usepackage{svg}
\usepackage{bigstrut}
\usepackage{wrapfig,lipsum,booktabs}
\usepackage{subcaption}
\usepackage{xfrac}
\usepackage{pdfpages}
\usepackage{listings}
\usepackage{xcolor}


\usepackage{listings}
\usepackage{color} %red, green, blue, yellow, cyan, magenta, black, white
\definecolor{mygreen}{RGB}{28,172,0} % color values Red, Green, Blue
\definecolor{mylilas}{RGB}{170,55,241}

\definecolor{codegreen}{rgb}{0,0.6,0}
\definecolor{codegray}{rgb}{0.5,0.5,0.5}
\definecolor{codepurple}{rgb}{0.58,0,0.82}
\definecolor{backcolour}{rgb}{1,1,1}

\lstdefinestyle{mystyle}{
    backgroundcolor=\color{backcolour},   
    commentstyle=\color{codegreen},
    keywordstyle=\color{magenta},
    numberstyle=\tiny\color{codegray},
    stringstyle=\color{codepurple},
    basicstyle=\ttfamily\footnotesize,
    breakatwhitespace=false,         
    breaklines=true,                 
    captionpos=b,                    
    keepspaces=true,                 
    numbers=left,                    
    numbersep=5pt,                  
    showspaces=false,                
    showstringspaces=false,
    showtabs=false,                  
    tabsize=2
}
\lstset{style=mystyle}


%date format
\def\mydate{\leavevmode\hbox{\twodigits\day.\twodigits\month.\the\year}}
\def\twodigits#1{\ifnum#1<10 0\fi\the#1}

\usepackage{indentfirst}
\setlength{\parindent}{1cm}

\makeatletter
\newcommand{\thickhline}{%
    \noalign {\ifnum 0=`}\fi \hrule height 2pt
    \futurelet \reserved@a \@xhline
}
\newcolumntype{"}{@{\hskip\tabcolsep\vrule width 2pt\hskip\tabcolsep}}
\makeatother
\newcolumntype{?}{!{\vrule width 2pt}}
%%DOC ENVIROMENT%%%%%%%%%%%%%%%%%%%%%%%
\begin{document}
%Title 
\begin{flushleft}%% left justification
	\textbf{\Large{MKC-PKS: Úkol č. 4}}\hfill Filip Paul\\
	\large{Multimediální služby v sítích IP \hfill\mydate}
\end{flushleft}
\section*{Zadání úkolu 1:}
Jaká je efektivita přenosu hovorového signálu (VoIP) v síti Ethernet při uvažování klasického
formátu PCM (vzorkování 8kHz, 8b/vzorek, bez komprimace)? Uvažujte tři různé rámcové
rychlosti: 40, 60 a 80 rámců/s. Přenos probíhá protokolem RTP.
Efektivitu vyjádřete jako podíl užitečného toku dat a celkového toku (vč. hlaviček) na linkové
vrstvě Ethernet. (2b)

\section*{Vypracování úkolu 1:}
\noindent Režie Ethernetového rámce je 144 bitů, režie RTP protokolu je 87 bitů.
Celková režie (H) tak činí 231 bitů.\\
Velikost přenášeného rámce (bez režie):\\
$L_{40}$ = 64\,kbit/f = 64kbit / 40 = 1600 bitů\\
$L_{60}$ = 64\,kbit/f = 64kbit / 60 = 1067 bitů\\
$L_{80}$ = 64\,kbit/f = 64kbit / 80 = 800 bitů\\

\noindent Efektivita E:\\\\
$E_{40} = \dfrac{L}{H+L} = \dfrac{1600}{231+1600} = 87.38\%$\\\\
$E_{60} = \dfrac{L}{H+L} = \dfrac{1067}{231+1067} = 82.2\%$\\\\
$E_{80} = \dfrac{L}{H+L} = \dfrac{800}{231+800} = 77.59\%$\\\\
	
\section*{Zadání úkolu 2:}
Uvažujte prioritní systém WFQ se třemi frontami. Plánovač vybírá v každém kole z první fronty 
n1 = 1 paket, z druhé n2 = 2 pakety a z třetí n3 = 5 paketů. Vstupní toky do jednotlivých front 
jsou R1 = 300Mb/s, R2 = 300Mb/s, R3 = 500Mb/s. Uvažujte stejně dlouhé pakety. Jaké budou 
výstupní toky z front, když celková kapacita výstupní linky je 1Gb/s? (1b)

\section*{Vypracování úkolu 2:}
\noindent Výstupní toky:\\
Fronta 1: $R_{out} = \dfrac{1}{1+2+5}\cdot 1Gb/s = 125Mb/s\rightarrow$ + 33Mbit z 2 + 42Mbit z 3 = 200Mbit/s \\\\
Fronta 2: $R_{out} = \dfrac{2}{1+2+5}\cdot 1Gb/s = 250Mb/s\rightarrow$ + 83 Mbit z 3 = 283Mbit/s $\rightarrow$  33Mbit/s pro 1; R = 300Mbit/s\\\\
Fronta 3: $R_{out} = \dfrac{5}{1+2+5}\cdot 1Gb/s = 625Mb/s\rightarrow$ zbyde 125Mbit (83Mbit do 2 a 42 do 1); R = 500Mbit/s\\\\
\clearpage

\section*{Zadání úkolu 3:}
Předpokládejte, že zdroj dat je připojen k omezovači typu Leaky/Token Bucket linkou 10Mb/s. 
Navrhněte přítok žetonů a velikost vědra tak, aby uživatel mohl bez omezení odeslat dávku 
nejvýše 5MB a pak byl jeho tok omezen na 1Mb/s. Velikost paketu předpokládejte 1000B. (2b)

\section*{Vypracování úkolu 3:}
\noindent Počet žetonů:\\\\
$r = \dfrac{Rs}{L} = \dfrac{1Mbit/s}{8000} = 125token/s$\\\\
velikost vědra:\\\\
$t_1 = \dfrac{5\,MB\cdot 8}{10Mbit/s} = 4s$\\\\
$b = \dfrac{R_1\cdot t_1}{L} -t_1 \cdot r =  \dfrac{10Mbit\cdot 4}{8000} -4 \cdot 125 = 4500b$


\end{document}