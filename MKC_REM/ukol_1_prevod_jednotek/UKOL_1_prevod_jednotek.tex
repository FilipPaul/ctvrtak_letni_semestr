%%PREAMBLE %%%%%%%%%%%%%%%%%%%%%%%%%%%%
\documentclass[10pt, a4paper]{article}% size of txt = 10pt
\usepackage[top= 2cm,
			bottom = 2cm,
			left = 1.7cm,
			right = 1.7cm,
			footskip = 0.5cm,
			headsep = 0cm,
			headheight = 0cm
					]{geometry}
\usepackage{amsmath} % math packages
\usepackage{amsfonts}% math packages
\usepackage{amssymb} % math packages
\usepackage{graphicx} %package for including graphics
\usepackage{array}
\usepackage[thinlines]{easytable}
\usepackage{float}
\usepackage[section]{placeins}
\usepackage[hidelinks]{hyperref}
\usepackage[shortlabels]{enumitem}
\usepackage{svg}
\usepackage{bigstrut}
\usepackage{wrapfig,lipsum,booktabs}
\usepackage{subcaption}
\usepackage{xfrac}
\usepackage{pdfpages}
\usepackage{listings}
\usepackage{xcolor}


\usepackage{listings}
\usepackage{color} %red, green, blue, yellow, cyan, magenta, black, white
\definecolor{mygreen}{RGB}{28,172,0} % color values Red, Green, Blue
\definecolor{mylilas}{RGB}{170,55,241}

\definecolor{codegreen}{rgb}{0,0.6,0}
\definecolor{codegray}{rgb}{0.5,0.5,0.5}
\definecolor{codepurple}{rgb}{0.58,0,0.82}
\definecolor{backcolour}{rgb}{1,1,1}

\lstdefinestyle{mystyle}{
    backgroundcolor=\color{backcolour},   
    commentstyle=\color{codegreen},
    keywordstyle=\color{magenta},
    numberstyle=\tiny\color{codegray},
    stringstyle=\color{codepurple},
    basicstyle=\ttfamily\footnotesize,
    breakatwhitespace=false,         
    breaklines=true,                 
    captionpos=b,                    
    keepspaces=true,                 
    numbers=left,                    
    numbersep=5pt,                  
    showspaces=false,                
    showstringspaces=false,
    showtabs=false,                  
    tabsize=2
}
\lstset{style=mystyle}


%date format
\def\mydate{\leavevmode\hbox{\twodigits\day.\twodigits\month.\the\year}}
\def\twodigits#1{\ifnum#1<10 0\fi\the#1}

\usepackage{indentfirst}
\setlength{\parindent}{1cm}

\makeatletter
\newcommand{\thickhline}{%
    \noalign {\ifnum 0=`}\fi \hrule height 2pt
    \futurelet \reserved@a \@xhline
}
\newcolumntype{"}{@{\hskip\tabcolsep\vrule width 2pt\hskip\tabcolsep}}
\makeatother
\newcolumntype{?}{!{\vrule width 2pt}}
%%DOC ENVIROMENT%%%%%%%%%%%%%%%%%%%%%%%
\begin{document}
%Title 
\begin{flushleft}%% left justification
	\textbf{\Large{MKC-REM: Úkol č. 1}}\hfill Filip Paul\\
	\large{Převody jednotek \hfill\mydate}
\end{flushleft}
	\section*{\Large Zadání:}
	Převeďte následující hodnoty 1; 2; 5 $\mu$W a 10; 20; 100 mV naměřené na kmitočtu 1,33 GHz pomocí
	spektrálního analyzátoru Rohde\&Schwarz FSL3. Převod proveďte na následující jednotky dBm
	a dB$\mu$V (všechny zadané hodnoty do obou jednotek).	
	\section*{\Large Vypracování:}
	Podle specifikace má R\&S FSL3 vstupní impedanci 50$\Omega$ nebo 75$\Omega$, kde právě 50$\Omega$ je
	defaultní hodnota. Všechny výpočty byly tedy prováděny pro vstupní impedanci 50$\Omega$.
	\\\\
	\textbf{Příklad výpočtu: 5$\mu W$ na dBm:}\\

	$P_{dBm} = 10\cdot log\left( \dfrac{P}{1\,mW} \right) = 10\cdot log\left( \dfrac{5\mu W}{1\,mW} \right) = -23.01\,dBm$
	\\\\\\
	\textbf{Příklad výpočtu: 5$\mu W$ na dB$\mu V$:}\\
	$P = \dfrac{V^2}{R} \rightarrow V = \sqrt{PR}$\\\\
	$V_{dB\mu V} = 20\cdot log \left( \dfrac{V}{1\mu V} \right) = 20\cdot log \left( \dfrac{\sqrt{P\cdot R}}{1\mu V} \right)
	=20\cdot log \left( \dfrac{\sqrt{5\mu W \cdot 50}}{1\mu V} \right) = 83.98\,dB\mu V$
	\\\\\\
	\textbf{Příklad výpočtu: 100mV na dBm:}\\
	$P = \dfrac{V^2}{R}$\\
	$P_{dBm} = 10\cdot log\left( \dfrac{P}{1\,mW} \right) = 10\cdot log\left( \dfrac{ V^2/R}{1\,mW} \right)
	=10\cdot log\left( \dfrac{(100\,mV)^2 \cdot 50}{1\,mW} \right) = -6.98\,dBm $\\
	\\\\\\
	\textbf{Příklad výpočtu: 100mV na dB$\mu$V:}\\
	$V_{dB\mu V} = 20\cdot log \left( \dfrac{V}{1\mu V} \right) = 20\cdot log \left( \dfrac{100\,mV}{1\mu V} \right)
	= 100\,dB\mu V$
	\\\\\\
	\textbf{Výsledky:}
	Všechny následující výsledky byly vypočítány přiloženým python scriptem, který si můžete
	zobrazit na mojem v mojem github repozitáři
	\href{https://github.com/FilipPaul/ctvrtak_letni_semestr/blob/main/MKC_REM/ukol_1_prevod_jednotek/README.md}{\color{blue} zde}.
	\\\\
	\noindent
	\begin{minipage}[t]{0.5\textwidth}
		\textbf {Převod výkon $\rightarrow$ dBm:}\\
		$1W \rightarrow 30.00dBm$ \\
		$2W \rightarrow 33.01dBm$ \\
		$5\mu W \rightarrow -23.01dBm$ \\
		
		\textbf {Převod výkon $\rightarrow$ dB$\mu $V:}\\
		$1W \rightarrow 136.99dB\mu V$ \\
		$2W \rightarrow 140.00dB\mu V$ \\
		$5\mu W \rightarrow 83.98dB\mu V$ \\
	\end{minipage}% % leave no gap
	\begin{minipage}[t]{0.5\textwidth}
		\textbf { Převod napětí $\rightarrow$ dB$\mu $V:}\\
		$10V \rightarrow 140.00dB\mu V$ \\
		$20V \rightarrow 146.02dB\mu V$ \\
		$100mV \rightarrow 100.00dB\mu V$ \\
		
		\textbf {Převod napětí $\rightarrow$ dBm:}\\
		$10V \rightarrow 33.01dBm$ \\
		$20V \rightarrow 39.03dBm$ \\
		$100mV \rightarrow -6.99dBm$ \\
	\end{minipage}
\end{document}