%%PREAMBLE %%%%%%%%%%%%%%%%%%%%%%%%%%%%
\documentclass[10pt, a4paper]{article}% size of txt = 10pt
\usepackage[top= 2cm,
			bottom = 2cm,
			left = 1.7cm,
			right = 1.7cm,
			footskip = 0.5cm,
			headsep = 0cm,
			headheight = 0cm
					]{geometry}
\usepackage{amsmath} % math packages
\usepackage{amsfonts}% math packages
\usepackage{amssymb} % math packages
\usepackage{graphicx} %package for including graphics
\usepackage{array}
\usepackage[thinlines]{easytable}
\usepackage{float}
\usepackage[section]{placeins}
\usepackage[hidelinks]{hyperref}
\usepackage[shortlabels]{enumitem}
\usepackage{svg}
\usepackage{bigstrut}
\usepackage{wrapfig,lipsum,booktabs}
\usepackage{subcaption}
\usepackage{xfrac}
\usepackage{pdfpages}
\usepackage{listings}
\usepackage{xcolor}


\usepackage{listings}
\usepackage{color} %red, green, blue, yellow, cyan, magenta, black, white
\definecolor{mygreen}{RGB}{28,172,0} % color values Red, Green, Blue
\definecolor{mylilas}{RGB}{170,55,241}

\definecolor{codegreen}{rgb}{0,0.6,0}
\definecolor{codegray}{rgb}{0.5,0.5,0.5}
\definecolor{codepurple}{rgb}{0.58,0,0.82}
\definecolor{backcolour}{rgb}{1,1,1}

\lstdefinestyle{mystyle}{
    backgroundcolor=\color{backcolour},   
    commentstyle=\color{codegreen},
    keywordstyle=\color{magenta},
    numberstyle=\tiny\color{codegray},
    stringstyle=\color{codepurple},
    basicstyle=\ttfamily\footnotesize,
    breakatwhitespace=false,         
    breaklines=true,                 
    captionpos=b,                    
    keepspaces=true,                 
    numbers=left,                    
    numbersep=5pt,                  
    showspaces=false,                
    showstringspaces=false,
    showtabs=false,                  
    tabsize=2
}
\lstset{style=mystyle}


%date format
\def\mydate{\leavevmode\hbox{\twodigits\day.\twodigits\month.\the\year}}
\def\twodigits#1{\ifnum#1<10 0\fi\the#1}

\usepackage{indentfirst}
\setlength{\parindent}{1cm}

\makeatletter
\newcommand{\thickhline}{%
    \noalign {\ifnum 0=`}\fi \hrule height 2pt
    \futurelet \reserved@a \@xhline
}
\newcolumntype{"}{@{\hskip\tabcolsep\vrule width 2pt\hskip\tabcolsep}}
\makeatother
\newcolumntype{?}{!{\vrule width 2pt}}
%%DOC ENVIROMENT%%%%%%%%%%%%%%%%%%%%%%%
\begin{document}
%Title 
\begin{flushleft}%% left justification
	\textbf{\Large{MKC-REM: Úkol č. 3}}\hfill Filip Paul\\
	\large{Měření intenzity el. pole pomocí spektrálního analyzátoru \hfill\mydate}
\end{flushleft}
	\section*{\Large Zadání:}
	Určete velikost intenzity elektrického pole v místě fázového středu antény na následujících
	kmitočtech 2400, 2450 a 2500 MHz? Vlastní měření bylo provedeno EMC spektrálním
	analyzátorem Rohde\&Schwarz ESPI7 a na těchto kmitočtech byly naměřeny následující údaje:
	-130 dBm, 8 dB$\mu$V a -92 dBm. Hodnoty anténního faktoru použité antény typu Bi-Log pro výše
	uvedené kmitočty jsou: 20, 14 a 12 dB/m. Dále byl zjištěn vlastní šum použitého spektrálního
	analyzátoru a dosahoval následujících hodnot: -23, -20, -18 dB$\mu$V.


	\section*{\Large Vypracování:}
	Podle datasheetu je možné na spekrálním anaylzátoru Rohde\&Schwarz ESPI7 nastavit vstupní impedanci na 50 nebo 75 $\Omega$.
	Pro výpočty jsem uvažoval vstupní impedanci právě 50 $\Omega$. Vzhledem k oblíbenosti převodů jednotek jsem 
	pro veškeré převody vytvořil python script, pomocí kterého jsou všechny převody prováděny. V následující tabulce
	jsou již převedené hodnoty. Z druhého sloupce tabulky pro f = 2400\,MHz je patrné, že naměřený signál
	je srovnatelný s vlastním šumem přístroje. Takže hodnoty v tomto sloupci asi nelze brát úplně vážně.
	V přiloženém python scriptu, který si můžete zobrazit i v mém GITHUB repozitáři
	\href{https://github.com/FilipPaul/ctvrtak_letni_semestr/blob/main/MKC_REM/ukol_3_SPEKTRAK/README.md}{\color{blue} zde}, je
	formou komentářů v kódu stručný popis postupu.

	 % Table generated by Excel2LaTeX from sheet 'List1'
	 \begin{table}[htbp]
		\centering
		  \begin{tabular}{|l|c|c|c|}
		  \hline
		  \textbf{f} & \textbf{2400 MHz} & \textbf{2450 MHZ} & \textbf{2500 MHz} \bigstrut\\
		  \hline
		  \textbf{Změřeno} & 0.07 $\mu$V &  2.51 $\mu$V &  5.62 $\mu$V \bigstrut\\
		  \hline
		  \textbf{ŠUM} & 0.07 $\mu$V &  0.10 $\mu$V &  0.04 $\mu$V \bigstrut\\
		  \hline
		  \textbf{SNR} & -0.01 dB &  28.00 dB &  42.99 dB \bigstrut\\
		  \hline
		  \textbf{SIGNAL} & -188.51 dBm &  -99.34 dBm &  -92.06 dBm \bigstrut\\
		  \hline
		  \textbf{INTENZITA} & -168.51 dBV/m &  -85.34 dBV/m &  -80.06 dBV/m \bigstrut\\
		  \hline
		  \end{tabular}%
		\label{tab:addlabel}%
	  \end{table}%
	 
\clearpage
\section*{calc.py}
\lstinputlisting[language=python]{calc.py}

\section*{UnitConverter.py}
\lstinputlisting[language=python]{UnitConverter.py}
 

\end{document}