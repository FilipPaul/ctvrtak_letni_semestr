%%PREAMBLE %%%%%%%%%%%%%%%%%%%%%%%%%%%%
\documentclass[10pt, a4paper]{article}% size of txt = 10pt
\usepackage[top= 2cm,
			bottom = 2cm,
			left = 1.7cm,
			right = 1.7cm,
			footskip = 0.5cm,
			headsep = 0cm,
			headheight = 0cm
					]{geometry}
\usepackage{amsmath} % math packages
\usepackage{amsfonts}% math packages
\usepackage{amssymb} % math packages
\usepackage{graphicx} %package for including graphics
\usepackage{array}
\usepackage[thinlines]{easytable}
\usepackage{float}
\usepackage[section]{placeins}
\usepackage[hidelinks]{hyperref}
\usepackage[shortlabels]{enumitem}
\usepackage{svg}
\usepackage{bigstrut}
\usepackage{wrapfig,lipsum,booktabs}
\usepackage{subcaption}
\usepackage{xfrac}
\usepackage{pdfpages}
\usepackage{listings}
\usepackage{xcolor}


\usepackage{listings}
\usepackage{color} %red, green, blue, yellow, cyan, magenta, black, white
\definecolor{mygreen}{RGB}{28,172,0} % color values Red, Green, Blue
\definecolor{mylilas}{RGB}{170,55,241}

\definecolor{codegreen}{rgb}{0,0.6,0}
\definecolor{codegray}{rgb}{0.5,0.5,0.5}
\definecolor{codepurple}{rgb}{0.58,0,0.82}
\definecolor{backcolour}{rgb}{1,1,1}

\lstdefinestyle{mystyle}{
    backgroundcolor=\color{backcolour},   
    commentstyle=\color{codegreen},
    keywordstyle=\color{magenta},
    numberstyle=\tiny\color{codegray},
    stringstyle=\color{codepurple},
    basicstyle=\ttfamily\footnotesize,
    breakatwhitespace=false,         
    breaklines=true,                 
    captionpos=b,                    
    keepspaces=true,                 
    numbers=left,                    
    numbersep=5pt,                  
    showspaces=false,                
    showstringspaces=false,
    showtabs=false,                  
    tabsize=2
}
\lstset{style=mystyle}


%date format
\def\mydate{\leavevmode\hbox{\twodigits\day.\twodigits\month.\the\year}}
\def\twodigits#1{\ifnum#1<10 0\fi\the#1}

\usepackage{indentfirst}
\setlength{\parindent}{1cm}

\makeatletter
\newcommand{\thickhline}{%
    \noalign {\ifnum 0=`}\fi \hrule height 2pt
    \futurelet \reserved@a \@xhline
}
\newcolumntype{"}{@{\hskip\tabcolsep\vrule width 2pt\hskip\tabcolsep}}
\makeatother
\newcolumntype{?}{!{\vrule width 2pt}}
%%DOC ENVIROMENT%%%%%%%%%%%%%%%%%%%%%%%
\begin{document}
%Title 
\begin{flushleft}%% left justification
	\textbf{\Large{MKC-REM: Úkol č. 2}}\hfill Filip Paul\\
	\large{Útlum filtru \hfill\mydate}
\end{flushleft}
	\section*{\Large Zadání:}
	Substituční měřicí metodou dle ČSN CISPR 17 byl proměřen vložný útlum fltru. Při měření byly
	použity následující měřicí přístroje: generátor Rohde \& Schwarz SML03 a spektrální analyzátor
	Rohde \& Schwarz FSL3.Výstupní výkon generátoru byl nastaven na 7 dBm Měření probíhalo ve
	dvou krocích dle obrázku a byly naměřeny následující hodnoty pomocí spektrálního analyzátoru na
	jednotlivých zadaných kmitočtech:


	\section*{\Large Zadání:}
	Následující tabulka obsahuje již přepočtené hodnoty napětí na výkon. Pro přepočet všech hodnot byl
	použit přiložený python script (pro lepší zobrazení můžete využít link na GITHUB 
	\href{https://github.com/FilipPaul/ctvrtak_letni_semestr/blob/main/MKC_REM/ukol_2_FILTR/README.md}{\color{blue} README})
	\\\\
	\noindent
	\begin{minipage}[t]{0.49\textwidth}
		\vspace*{-2cm}
		\textbf {Převod $dB\mu V$ $\rightarrow$ dBm pro 113.6$dB\mu V$:}\\\\
		$V = 10^{\dfrac{dB\mu V}{20}}\cdot 1\mu V = 10^{\dfrac{113.6}{20}}\cdot 1\mu = 0.4786 V$\\\\
		$PdBm = 10log\left( \dfrac{V^2}{R\cdot1\,mW} \right) = 10log\left( \dfrac{0.4786^2}{50\cdot1\,mW} \right)\\
		PdBm =6.61\,dBm $\\\\
		\textbf {Výpočet útlumu: (pro první řádek)}\\
		$L = PdBm_A - PdBm_B = 6.1\,dBm - 5.8\,dBm = 0.3\,dB$ 

	\end{minipage}% % leave no gap
	\hfill
	\begin{minipage}[t]{0.49\textwidth}
		\begin{tabular}{?c?c?c?c?}
		\thickhline 
		freq & krok A & krok B & útlum L   \bigstrut\\
		\thickhline 
		10kHz & 6.1dBm & 5.8dBm & 0.30dB \bigstrut\\
		\hline
		20kHz & 5.8dBm & 4.21dBm & 1.59dB \bigstrut\\
		\hline
		50kHz & 6.8dBm & -12.9dBm & 19.70dB \bigstrut\\
		\hline
		100kHz & 6.61dBm & -27.49dBm & 34.10dB \bigstrut\\
		\hline
		200kHz & 5.3dBm & -30.6dBm & 35.90dB \bigstrut\\
		\hline
		300kHz & 4.7dBm & -22.89dBm & 27.59dB \bigstrut\\
		\thickhline  
		\end{tabular}%
		\label{tab:addlabel}%
	\end{minipage}
	\\
	\section*{\Large Python Script}
	\lstinputlisting[language=python]{calc.py}

 

\end{document}